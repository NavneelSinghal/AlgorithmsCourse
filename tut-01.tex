% CUSTOM TEMPLATE FOR SOLUTIONS STARTS
\documentclass[answers]{exam}
 \usepackage{graphicx}
 \usepackage{float}
 \usepackage{amsmath}
 \usepackage{amssymb}
 \usepackage{amsfonts}
 \usepackage{framed}
 \usepackage{algorithmicx}
 \usepackage[noend]{algpseudocode}
 \newcommand{\ans}[1]{\begin{framed}{\textbf{Answer:} #1}\end{framed}}
 \newcommand{\sol}{\uplevel{\textsc{Solution:}}}
 \newenvironment{answer}{%
     \renewcommand{\solutiontitle}{\noindent\textbf{Answer:}\enspace}
     \begin{solution}
     }{%
     \end{solution}
     \renewcommand{\solutiontitle}{\noindent\textbf{Solution:}\enspace}
 }
 \newenvironment{claim}[1]{\par\noindent\underline{Claim:}\space#1}{}
 \newenvironment{claimproof}[1]{\par\noindent\underline{Proof:}\space#1}{\hfill $\blacksquare$}
 \newenvironment{proof}[1]{\par\noindent\underline{Proof:}\space#1}{\hfill $\blacksquare$}
 
 % CUSTOM TEMPLATE FOR SOLUTIONS ENDS

% First we setup the header and footer
\pagestyle{headandfoot}
\runningheadrule
\runningfootrule
\header{COL351: Analysis and Design of Algorithms (CSE, IITD, Semester-I-2020-21)}{}{Homework-00}
\footer{}{\thepage  \, of \numpages}{}
 
% We want the points for each question displayed on the left
%\pointname{points}
%\pointsinmargin
 
% Automatically total the points - make sure to compile TWICE
\addpoints
 
\begin{document}


\begin{questions}

    \question
    \begin{solution}
    \begin{claim}
        Let $P(k)$ denote the statement `for $2^k \le n < 2^{k + 1}$, we have $T(n) = 2^k$'.
        Then, $\forall \,\, k \in \mathbb{N}\cup \{0\}$, $P(k)$ is true.
    \end{claim}
    \begin{proof}
    Proof:
        We perform induction on $k$.\\
        Base case: $k = 0$. In this case, $1 \le n < 2$, so $n = 1$, and since $T(1) = 1 = 2^0 = 2^k$, we are done.\\
        Inductive hypothesis: $P(i)$ is true for some $i \in \mathbb{N} \cup \{0\}$.\\
        Inductive step: Suppose $2^{i + 1} \le n < 2^{i + 2}$.

\begin{enumerate}
\item $n$ is even. In this case, $T(n) = 2T(n / 2)$. Since $2^i \le n / 2 < 2^{i + 1}$, invoking the inductive hypothesis, we have $T(n / 2) = 2^{i}$, so $T(n) = 2^{i + 1}$.
\item $n$ is odd. Here $n > 2^{i + 1}$, so $T(n) = T(n - 1) = 2T(\frac{n - 1}{2})$. Note that $2^i \le \frac{n - 1}{2} < 2^{i + 1}$, so $T(\frac{n - 1}{2}) = 2^i$, so $T(n) = 2^{i + 1}$.
\end{enumerate}
        
        Since in either case, $T(n) = i + 1$, $P(i) \implies P(i + 1)$.\\
        Thus by the principle of mathematical induction we are done.
    \end{proof}
\end{solution}

\question

    \begin{solution}
        I claim that the answer is false.\\
        I'll show that for any constant $c > 0$, there exists a positive integer $n_0$ such that $g(n) > c f(n)$ for all $n > n_0$.
        Note that if $n > 2c$, we have $\frac{n}{2} \cdot 3^n > c \cdot 3^n$, and for $n > \log_{\frac{3}{2}} (20c)$, we have $\frac{n}{2} 3^n > 10c \cdot n \cdot 2^n$.\\
        Thus for $n > \max(2c, \log_{\frac32}(20c))$, we have, by adding the above inequalities, $n \cdot 3^n > c \cdot \left(10n2^n + 3^n\right)$, as needed.
    \end{solution}

\end{questions}
\end{document}

